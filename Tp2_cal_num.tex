\documentclass[]{article}
\title{TD/TP Calcul Numérique}
\author{Diop}
\usepackage[utf8]{inputenc}
\begin{document}
\section{Exercice 6 Rappel de $ scilab$}
\begin{itemize}
\item {Question 1. \\}
$ x = rand(3,1) $ est vecteur de taille 3 lignes et 1 colonne.\\
\item {Question 2. \\}
$ y = rand(4,1) $ est vecteur de taille 4 lignes et 1 colonne.\\

\item {Question 3. \\}
Les opérations $ z = x + y$ et $ s = xy$ ne peuvent pas se réaliser car la taille du vecteur x qui est 3 lignes et 1 colonne est différente de celle de y qui est 4 lignes et 1 colonne.\\

\item {Question 4. \\}
Avec la fonction size(), on a calculé la taille de x qui est de  3 lignes et 1 colonne et celle de y qui est de 4 lignes et 1 colonne.\\

\item {Question 5. \\}
La fonction $norm()$ de $scliab$ nous permet de calculer la norme de x qui est : $a$.\\

\item {Question 6. \\} 
La matrice $ A = rand(4,3) $ est une matrice de 4 lignes et de 3 colonnes.\\

\item {question 7. \\}
A' est la transposée de la matrice A, donc elle est de 3 lignes et de 4 colonnes.\\

\item {Question 8. \\}
Soient $ A = rand(4,4) $ et $ B = rand(4,4) $.\\
On effectue les opérations élémentaire des deux matrices A et B.\\
on a :\\
$ det(A) = $;\\
$ det(B) = $;\\
$A + B = $;\\
$A*B = $;\\

\item {question 9. \\}
On calcul le conditionnement de la matrice A.\\
$cond(A) = $;\\

\section{Exercice 7  Matrice $ random$ et problème "jouet"}

\begin{itemize}
\item {Question 1. \\}
On écris une matrice A de taille $3*3$ : $ A = rand(3,3)$.\\

\item {question 2. \\}
On écrit un vecteur $xex$ dans $R^{3}$ avec la fonction $rand()$ : $ xex = rand(3,3)$.\\
on vérifie bien avec la fonction $size()$ que  $ xex $ est un vecteur colonne car sa taille est : $ 3*1$.\\

\item {question 3. \\}
On écrit $b$ le produit de $ A*xex $ : $ b = A*xex$.\\
Avec la fonction $ size(), $on voit que $b$ est vecteur colonne car $3*1$.\\ 
\end{itemize}
\end{itemize}
\end{document}